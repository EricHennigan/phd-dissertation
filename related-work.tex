

\begin{comment}
* 2009-xss-survey
  client-side solutions:
  - ismail et al. 2004 (rewrite browser requests, detect reflected XSS)
  - Kirda et al. 2006 (firewall intercept HTTP, block/allow connection)
  - Maone 2007 (NoScript, disables Java ActiveX + JS)
  - Vogt et al. 2007 (dyn+static data tainting + user policy)
  - Microsoft 2007 http-only cookie
  - Jim et al. 2007 BEEP (whitelisting + sandbox)
  - Yu et al. 2007 CoreScript (edit automata inserted into JS)
  - Chugh et al. 2009 staged info flow, lazy static analysis

  Server-side solutions:
  - Kruegel and Vigna 2003, anomaly detection
  - Nguyen-Tuong et al. 2004 dyn data taint in PHP
  - Huang et al. 2004 WebSSARI, instrumentation of sanitization
  - Pietraszek and Berghe 2005, CSSE (tracks user-generated string for mysql query)
  - Haldar et al. dynamic tainting in JVM
  - Xie and Aiken 2006, PHP calls to SQL
  - Jovanovic et al. 2006 Pixy, static data flow PHP
  - Reis 2006 BrowserShield, pass all JS through filter (JS rewriting)
  - Bisht and Venkatakrishnan 2008, shadow HTTP request
  - Phung 2009, inlined reference monitors

  Hybric solutions:
  - Nadji 2009, document structure integrity (pseudo random number sequence on tags)
  - Scott and Sharp 2002, Security policy description language web app firewall
  - Huang 2003, WAVES, fault-injection to find SQL injection, behavior monitoring to find XSS
  - Kc 2003, randomize function calls with key to prevent injection
  - Lucca 2004, automated test find user input to execute XSS strings
  - Kals 2006, SecuBat, tests script injection
  - Johns 2008, traffic monitor
  - Wassermann and Su 2008, formal language model find weak input validation
  - Maffeis and Taly 2009, secure subset of JS (analyzed FBJS AdSafe)

* 2012_crisis
  suggested by reviewers:
  - Austin, Flannagan ``Multiple Facets for Dynamic Information Flow''
  - Sabelfeld, Li, Russo ``Implicit flows in malicious and nonmalicious code''

* Other
  - Seth Just Information flow analysis for JS has a good discussion about control flow structures.

There are two outstanding problems: 1. label creep, 2. can't track passive indirect flows.

%http://www.cse.psu.edu/~tjaeger/cse598-f11/docs/sm-jsac03.pdf
%http://www.cse.chalmers.se/~andrei/psi09.pdf

\end{comment}

\chapter{Related Work}\label{ch:related-work}

The amount of research on preventing cross-site scripting underscores its importance.
This chapter shows how our work relates to the increasingly active field of information flow security.

Several other works on information flow control in JavaScript, such as that by Hedin and Sabelfeld~\cite{hedin.sabelfeld+12} and Austin and Flanagan~\cite{austin.flanagan+09,austin.flanagan+10,austin.flanagan+12}, influenced the design and implementation of \JitFlow.
Kerschbaumer~et~al.\cite{kerschbaumer.etal+13} uses the interpreter portion of \JitFlow\ as part of a comprehensive solution that tracks scripting-exposed subsystems in WebKit, including JavaScript VM, the DOM, and user generated events.
We think that all of the approaches mentioned below can immediately benefit from increased performance via our contribution of techinques to implement information flow tracking for JIT compilers.

%The survey paper by Sabelfeld and Myers~\cite{sabelfeld.myers+03} summarizes research on language-based information flow up until 2003.
%Most of those efforts focused on static analysis for information flow control in strongly typed languages.

\section{Foundations of Information Flow}

%In 1977, Denning and Denning~\cite{denning-cert} introduced a Pascal compiler that uses static program analysis and a security lattice over principals to certify program non-interference at compile time.
%This compiler implements a security lattice~\cite{denning-lattice} over principals.
%If the compiler detects an implicit information flow in a program, the program fails certification.
%In contrast, our supporting JavaScript VM performs information flow propagation at runtime and ensures that code cannot send traffic over the network if it violates a developer-specified policy.

The development of Information Flow dates back to the late 1970's, and arises out of a desire to trace and detect information leaks that can occur in computer systems.
In 1976, Denning introduced ``A Lattice Model of Secure Information Flow''~\cite{denning+76} that permits a concise formulation of security requirements through the mathematical relations on a partially ordered set of security principals.
The lattice model provides a unifying view of all systems that restrict information flow, overcoming undecidable analysis present in security mechanisms based on access control~\cite{lampson+74}.
Shortly following this insight, Denning and Denning~\cite{denning.denning+77} introduced a Pascal compiler that exploited the properties of the security lattice to verify secure information flow through a program.
A program fails certification when the static analysis phase of the compiler detects an implicit information flow.

In 2001, the programming language Jif (Java Information Flow), introduced by Meyers et al.\cite{jif} proved the feasibly of tracking information flows within real programs.
The Jif interpreter introduces a security-type system that allows annotation of Java types with confidentiality labels that refer to variables of the dependent type \texttt{label}~\cite{sabelfeld.myers+03}.
%This feature allows label objects to be used as both first-class values and as labels for other values.
Jif extends the lattice model with a system of decentralized labels~\cite{myers.liskov+00}.
\JitFlow\ borrows this approach for dealing with the multitude of principals that require separate and distinct representation within a web page.

Both of these works demonstrate the ability of static typing systems to verify secure information flow.
\JitFlow\ incorporates these insights into its labeling mechanism, adjusting as necessary for a dynamically typed language.

\section{Information Flow in Other Languages}

Chandra and Franz~\cite{chandra.franz+07} present an information flow framework for the Java~VM combining static and dynamic techniques.
A static analysis annotates paths of information flow within the resulting bytecode.
At runtime, the VM uses these annotations to maintain labels on variables lying in \emph{non}-executed control flow paths.
Unlike other approaches that freeze policies at compile time, their system preserves enough separation between the policy and enforcement mechanisms that policy changes can be dynamically updated during runtime.
\JitFlow\ shares a focus on the bytecode, which enables lower-level instrumentation and analysis.

\section{Information Flow Tracking in JavaScript}

Unlike statically typed languages such as Java, JavaScript code benefits from dynamic analysis during program execution.
In addition to loading code at the browser's request, JavaScript allows and frequently uses the \code{eval} function~\cite{nikiforakis.etal+12}, which converts strings into code.
As a result, static analysis techniques can never analyze all code before execution.
Unfortunately, dynamic program analysis has drawbacks, too.
Unlike static analysis, it both adds to the execution time and restricts analysis to properties of actually executed code paths.
The latter prevents a single execution of a dynamic analysis from determining passive implicit flows.
Consequently, other research also exhibits tracking up only up to active implicit flows.

\subsection{Source Rewriting}

Yu~et~al.\cite{yu.etal+07} use rewriting to force all untrusted JavaScript code through an instrumenting filter that identifies relevant operations, modifies questionable behaviors, and prompts the user (with a web page viewer) for decisions on how to proceed when appropriate.
A separate policy then ensures that the code behaves in a controlled manner, even permitting self-modification at runtime.

Chudnov and Naumann~\cite{chudnov.naumann+10} implement runtime-inline monitoring, based on its applicability to dynamically-typed, interpreted languages, featuring \code{eval}, such as JavaScript.
They argue that inlining should occur at the source code level, because of the widespread practice of delivering JavaScript source code to browsers and the non-portability of any VM-level implementation across browsers.
Their proposal relies on the JIT to maintain acceptable performance, but they provide no implementation, only a proof of correctness for the inlined monitor.
\FlowCore\ modifies the VM-runtime in order to maintain performance during a time when browser vendors heavily market their execution speend.
Additionally, their proof rests on a small abstract syntax that does not fully represent the complexity of JavaScript.

Jang~et~al.\cite{jang.etal+10} employ a JavaScript rewriting-based information flow engine to document 43 cases of history sniffing within the Alexa~\cite{alexa} Global Top 50,000 sites.
Their framework invokes a rewrite function on JavaScript code and encapsulates it into a monitored closure.
The interpreter invokes this function on JavaScript code before delivering it to the bytecode compiler.
Although rewriting the source can instrument policy enforcement mechanisms, their current implementation fails to detect implicit information flows.
They give no performance numbers, but we reason that these closures incur a high memory and function call overhead, something that \JitFlow\ seeks to prevent by operating at the instruction level.
%As part of a large-scale investigation into privacy-violating information flows that occur on the Alexa global top 50,000 websites, Jang~et~al.~\cite{jang.etal+10} implemented a rewriting-based JavaScript information flow engine within the Chrome browser.
%The system uses a source-to-source rewriting approach that injects taints, propagates them, and blocks tainted flows within the rewritten code, with an average execution overhead between 2.5$\times$ and 3$\times$.
%In contrast to \FlowCore, their system required understanding only the browser's AST data structures, and none of the complexity of the JavaScript runtime, and therefore tracks flows at a coarser granularity.
%Nevertheless, they detected four types of privacy-violating flows, cookie stealing, location hijacking, history sniffing, and behavior tracking, used by very popular sites (Alexa top-100) that exfiltrate information about users' browsing behavior.
%Because of access control restrictions on the History object, determining sites a user has visited can only be accomplished via an implicit control flow such as that in \autoref{list:sniffPassword}.
%Most websites that contained such flows were attempting to discern whether a user has visited pages of a competing site.


Magazinius~et~al.\cite{magazinius.etal+12} also employ the source-rewriting techinque that operates on-the-fly with an overhead between 2$\times$ and 3$\times$.
Their work inlines dynamic information flow monitors every time the interpreter evaluates a code string, which pertains to all JavaScript applications because they ship as source.
The monitors implement shadow variables that track the security label of the original in addition to a hidden shadow variable for tracking the program counter.
They demonstrate satisfaction of non-intereference property even for the dreaded \code{eval} statement.
The inlining technique remains advantageous in that it requires no modifications to the hosting environment.

\subsection{Control Flow Stack}

\JitFlow\ implements the control flow stack as a runtime shadow stack, which records the history of the labels attached to the program counter at each control flow branch.
Many researchers have used the runtime shadow stack technique to secure program execution~\cite{abadi.etal+09, frantzen.shuey+01, prasad.chiueh+03}.
It has also been successfully used in other information flow research~\cite{lam.chiueh+06}.
The \JitFlow\ implementation extends this research by instrumenting \emph{explicit} instructions for manipulating the shadow stack into the existing instruction stream.
After extensive literature review, we could not find any publications that introduce instructions for maintaining a runtime shadow stack data structure.
Indeed, we could find no authors which address these important details so vital to implementors.

Vogt~et~al.\cite{vogt.etal+07} modified an earlier version of SpiderMonkey to monitor the flow of sensitive information in the Mozilla web browser with dynamic data tainting.
Their system explicitly identified data sources and sinks within the Firefox browser.
Before execution, every script undergoes a static data flow analysis that simulates the VM operations on an \emph{abstract stack}, to determine existence of information leaks.
It initializes taint by marking sensitive data at each source and then tracks accesses dynamically.
Their framework handles control structures such as \code{throw} and \code{try} conservatively, by statically marking all variables within that function as tainted.
By monitoring the browser's data sinks, it detects when a script attempts to transfer tainted data to a third-party.
Although the tainting mechanisms in this work closely parallel our own, we incorporate a \emph{runtime} stack that allows for a more precise analysis about implicit flows which \emph{actively} occur.
\JitFlow's labeling system also represents more security principals than simple data tainting.

\subsection{Staged Analysis}

Chugh~et~al.\cite{chugh.etal+09} attack the problem of dynamically loaded JavaScript staging the information flow analysis.
Their approach statically computes an information flow graph for all available code, leaving ``holes'' where code might appear at runtime.
This technique separates programs into statically analyzable components and parts that require dynamic analysis at runtime.
Whenever new code becomes available, the browser subjects it to a static analysis that produces a subgraph of information flows.
When the new subgraph merges with the current information flow graph, the system performs residual checks to ensure that the combined result cannot violate existing policy constraints.
They also introduce a new policy language to the existing babel of languages used for web development.
In contrast, \FlowCore\ avoids delaying code execution and shifts analysis of information flows to runtime, enabling the developer to write application-specific policies in JavaScript itself.

Dhawan and Ganapathy~\cite{dhawan.ganapath+09} extended the approach to detect violating flows in browser extensions written in JavaScript.
Their system, called Sabre (Security Architecture for Browser Extensions), associates each in-memory JavaScript object with a label that determines whether the object contains sensitive information.
As with \JitFlow, labels propagate with modification of the object.
The browser raises an alert whenever the program accesses a secure object in an unsafe way (e.g. writes it to a file or the network).

Just~et~al.\cite{just.etal+11} improve on Vogt and Dhawan's approaches by adding support for control dependences created by unstructured control flow (\code{break} and \code{continue} statements) to the analysis of implicit indirect flows.
\JitFlow's integration with the JIT compiler achieves the same analysis with better performance due to a lower level implementation.


\subsection{Taint Tracking, Secure Multi-Execution, and Isolation}

Taint tracking approximates information flow security, but limits analysis to explicit flows.
The omission of implicit flows has two advantages.
First, the taint tracking overhead remains lower since it performs less tracking.
Enck~et~al.\cite{enck.etal+10}, for example, report an average overhead of 14\% with their taint tracking solution for Android.
Second, full tracking of implicit information flows requires static analysis~\cite{denning.denning+77,myers+99} or halting execution for some flows~\cite{austin.flanagan+09,austin.flanagan+10}.

Several researchers independently developed a dynamic execution technique known as Secure Multi-Execution (SME).
By evaluating all branches, SME prevents all explicit and implicit flows from occurring without the need to handle implicit indirect flows specially, e.g., via static program analysis.
Capizzi~et~al.\cite{capizzi.etal+08} execute a second copy of the Firefox browser and substitute inputs so that the two copies follow the same execution paths, one with public data and one with private data.
By limiting SME to the Javascript engine alone, Groef~et~al.\cite{groef.etal+12} lower the execution overhead in a project they call FlowFox.
Devriese and Piessens~\cite{devriese.piessens+10} formalize the SME technique and prove strong soundness and precision guarantees for noninterference.

Nevertheless, SME unfortunately suffers from high overheads in both time and space.
FlowFox, for instance, roughly doubles performance on Google's V8 benchmarks, but uses only two security principals (representing public and private).
Austin and Flanagan~\cite{austin.flanagan+12} use ``faceted values'' to optimize SME, but still note that a webpage with $n$ principals needs up to $2^n$ executions.

A number of researchers have evaluated isolation and sandboxing as a defense against XSS and other browser attacks.
Grier~et~al.\cite{grier.etal+08} built the OP browser which combines formal methods with operating system design principles.
It partitions the browser into subsystems with simple interactions and uses information flow to analyze attacks.
Nadji~et~al.\cite{nadji.etal+09} combine randomization of web content with runtime tracking to ensure that untrusted content injected into a page cannot be syntactically isolated from its surrounding content.
The strength of this approach, called document structure integrity, lies in its ability to prevent XSS attacks not based on JavaScript.
In contrast, \JitFlow\ does not address information flows encoded within the structure of host provided objects~\cite{russo.etal+09}.
AdSafe~\cite{adsafe}, Caja~\cite{caja}, and FaceBook~JS~\cite{facebookjs} exemplify a different approach focused on limiting a JavaScript programs capabilities, rather than identifying untrusted source code and isolating it during execution.

\subsection{Type-Checking JavaScript for Information Flow}

Many researchers give type systems intended to analyze JavaScript programs for information flow security.
\JitFlow\ forgoes formalized verification in a practical effort to target adoption of our work by developers focused on security debugging rather than end users.
%The ability (and business encouragement) to include third part scripts in a web mashup makes finding solutions to the \emph{label creep} problem~\cite{sabelfeld.myers+03} even more urgent than it historically has been.

Austin and Flannagan, in conjunction with Mozilla, promote sparse labeling techniques intended to decrease memory overhead and increase performance~\cite{austin.flanagan+09} and provide a formal semantics for \emph{partially leaked} information~\cite{austin.flanagan+10}.
Hedin and Sabelfeld~\cite{hedin.sabelfeld+12} provide Coq-verified formal rules that cover object semantics, higher-order functions, exceptions, and dynamic code evaluation, powerful enough to support DOM functionality.
Efforts along this line of research typically cover a core of the JavaScript specification, and have not seen wide-spread adoption.

\subsection{Formalization}

We currently do not have a formal proof that our framework can guarantee non-interference security.
Although some researchers have worked toward providing a formalization of JavaScript semantics~\cite{yu.etal+07, herman.flanagan+07, maffeis.etal+08, guha.etal+10} on which such a proof could be based, we did not find any that were readily suitable for creating such a proof.
These formalizations suffer from being incomplete with respect to all the features of JavaScript or are only available in paper form.
Tackling such a drawback will require much future work to bring these efforts into a state where they can be easily used by implementors as a verification framework within an automated proof system.
We eagerly await further research in this direction, so that we may identify and fix any bugs within our approach.

\section{First-Class Labels}

The difficulty of introducing information flow security into large bodies of existing code without developer assistance has been a long standing problem in the field~\cite{sabelfeld.myers+03}.
To the best of our knowledge, no other work incorporates a first-class labeling system into a dynamically typed programming language.
Instead other research on language-based information flow specific to JavaScript relies on automatic labeling frameworks that seek to provide end-users with secure browsers and minimize developer involvement.
But the first-class label feature allows the developer to construct label objects, apply them to label other program values, compose them together, and use them as part of natively programmed policy functions, within a security testing environment.
The first-class labels API moves beyond implementation of an information flow tracking engine to reflect portions of the labeling engine into the JavaScript environment, to enable targeted security debugging.

Li and Zdancewic~\cite{li.zdancewic+06} present a security sublanguage that expresses and enforces information-flow policies in Haskell.
Their implementation supports dynamic security lattices, run-time code privileges, and declassification without modifications to Haskell itself.
The type-checking proceeds in two stages: (1) checking and compilation of the base language followed by (2) checking of the secure computations at runtime just prior to execution of programs written in the sublanguage.
In contrast, our work presents extensions to an existing JavaScript environment and does not require rewriting of existing programs into a secure sublanguage.

\section{Just-In-Time Compilation}

\JitFlow\ leverages existing architecture practices common among JIT compilers for JavaScript code, so that the techniques used in its development remain applicable to other JavaScript engines.
Deutsch and Schiffman~\cite{deutsch.schiffman+84} performed early work on JIT compilation for the Smalltalk~80 system.
Aycock~\cite{aycock+03} gives a concise survey covering state-of-the-art developments in JIT compilation.
More recent advances in JIT compilation specifically in JavaScript engines were made by Gal~et~al.\cite{gal.etal+09} and Hackett and Guo~\cite{hackett.guo+12}.
To the best of our knowledge, \JitFlow\ marks the project to incorporate information flow into a JIT compiler.

\section{Policy Enforcement}

The Browser-Enforced Embedded Policy (BEEP) project~\cite{jim.etal+07} introduced the idea of allowing a webpage to specify which scripts to trust, using the browser itself to filter out entire scripts.
The framework hashes the source of each script and refers to a whitelist to determine the legitimacy of a script before executing it.
A website author must place this whitelist in the \texttt{<head>} portion of a webpage, so that the browser can load it before executing any JavaScript that might change the list.
Rather than focusing on the legitimacy of the script itself, \FlowCore\ preserves the flexibility of executing all scripts as long as they do not incur an information flow violation, enabling business to continue including dynamically delivered advertisements.

Meyerovich and Livshits introduce an aspect oriented framework named \textsc{ConScript}~\cite{meyerovich.livshits+10} that supports weaving specific security policies with existing web applications.
Using their framework, web authors wrap application code with fine-grained, application-specific security monitors specified in JavaScript, and enforced by a visitor's browser.
They also provide a type-checker that verifies that policies do not accidentally contain common implementation bugs and show how to automatically generate a policy via static analysis of server-side code or runtime analysis of client-side code.
Their system supports aspect wrapper functions around arbitrary code, while \FlowCore\ focuses on monitoring network traffic.
Although they define a policy specification framework that can refer to the browser objects exported to JavaScript runtime, it remains incapable of specifying a non-interference policy, so it cannot detect passive implicit information flows.
An aspect oriented approach cannot detect and prevent implicit information leaks that occur due to control-flow transfers, as exhibited in \autoref{list:sniffPassword}.

\section{Mashup Security}

Crites~et~al.\cite{crites.etal+08} have proposed ``OMash'', a mechanism which secures communication between scripts in a mashup, and overcomes the tradeoff between security and functionality imposed by the Same Origin Policy.
OMash treats web pages as program objects and restricts communication to declared public interfaces.
By abandoning the SOP for controlling DOM access and cross-domain data exchange, it avoids the SOP's vulnerabilities.

Barth~et~al.\cite{barth.etal+08} examine the existing \code{postMessage} API, that uses the ability of one frame to navigate another, providing a communication channel that bypasses the SOP, which restricts manipulation of objects across frame boundaries.
They analyze existing attacks that occur when using a frame's location URL as a communication channel, and propose extending the API to allow the sender of a message to specify the recipient.
