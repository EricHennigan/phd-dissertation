
\chapter{JavaScript Feature Catalog}

\section{Data Structures}
\subsection{Objects}
\subsection{First-Class Functions}
\subsection{Built-ins}
\subsubsection{Arrays}
\subsubsection{Date}
\subsubsection{Interned Strings (And Other Things)}

\section{Control Flow}
\subsection{Function Calls}
\subsection{Early Return}
\subsection{Break and Continue}
\subsection{Exceptions}
\subsection{Eval}


 - what to do with arrays, or does this fit better in design considerations?
   : Can coalesce labels on arrays?
   : label bounds checking? 
 - obj literals
   how they interact with obj poisoning attack
 - retrieval
   indexing syntax [] vs .
   prototype chain
 - Seth Just Information flow analysis for JS has a good discussion about control flow structures.
 - early returns (out of nested loops)
 - break/continue (in nested loops)

%%%%%%%%%%%%%%%%%%%%%%%%%
% From 2011 acsac.tex

Although dynamically typed languages present many difficulties for ensuring information flow, we find that JavaScript uses structured control flow constructs.
As a result, we can use a static analysis which calculates exactly at which points our security instructions should be inserted into the instruction stream.
This modification allows our system to prevent implicit information flow leaks by tracking the security label of the program counter at runtime, using the control flow stack.
%Because local variables cannot leak information, allowing them to automatically upgrade still complies with the intention of non-interference security~\cite{goguen1982security}.
%Variables which can leak information are subject to the \textit{no sensitive upgrade} check (see Section~\ref{sec:sensitive-upgrade-check}), and are prevented from automatic upgrade.
%Using these principles, we demonstrate the labeling strategy for several control flow structures defined in the JavaScript language~\cite{ecma}.

\subsection{Conditional Branches}
Conditional branches begin with a \texttt{DUP\_CFLABEL} instruction that marks the beginning of a secure code region by cloning the current pc-label.
The conditional value itself may be the result of an arbitrary expression, which could include function calls and shortcut evaluation of logical operators.
Consequently, its label is not predictable at compile time, so we emit a \texttt{JOIN\_CFLABEL} instruction immediately following the conditional evaluation.
This instruction upgrades the top of the control flow stack using the label of the conditional value computed at runtime.
When either side of the conditional branch finishes executing, a \texttt{POP\_CFLABEL} instruction at the control flow join restores the pc-label to its state before the branch was encountered.

\subsection{Loops}

Because of the implied backwards branch, loops require more care than conditional branches.
Prior to entering the loop a \texttt{DUP\_CFLABEL} instruction clones the current pc-label.

Again, because the condition is a runtime evaluated expression, only a dynamic analysis can identify the correct label to apply to the loop body.
The possibility of an earlier iteration influencing a later iteration complicates the situation.
Our implementation emits a \texttt{JOIN\_CFLABEL} instruction at the end of the conditional, despite the fact that this forces the current pc-label to be re-upgraded at each iteration.
Upon exiting the loop, a \texttt{POPJ\_CFLABEL} instruction restores the pc-label to its state before the loop was encountered.

One caveat of this solution is that it allows a monotonically increasing label on the loop body.
It is unfortunately possible that later iterations may carry a higher pc-label than earlier iterations, even when these iterations do not influence each other.
For example, an array might contain a sequence of completely unrelated values, each cloaked with a different security label.
When looping over such a construction, our implementation does not downgrade the loop context, even if independence of iterations could be proven.

\subsection{Break and Continue}
\label{sec:break-and-continue}
JavaScript allows the \texttt{break} and \texttt{continue} statements to specify which loop they apply to.
This complicates the maintenance of a control flow stack, because such statements may jump out of an arbitrarily nested loop.
Such jumps can bypass the normal exit criteria of nested loops, thereby causing the control flow stack to be out of alignment at the target location.
To maintain correct runtime-behavior we must handle this issue by ensuring that the instruction emitter generates the correct number of control flow pops for any nested loops that are exited.
A static analysis in the parser performs this computation and provides the resulting number as a parameter to the \texttt{POPJ\_CFLABEL} instruction.

Loop index variables, which are commonly bound to the function scope\footnote{JavaScript binds variables declared with \texttt{var} to the function scope, and did not have a block level scope prior to the introduction of \texttt{let} in version~1.7.} can be used for an implicit information leak if they are not all upgraded to the current pc-label at the time a \texttt{break} or \texttt{continue} statement is encountered.
Our system prevents potential implicit information leaks by emitting a \texttt{POPJ\_CFLABEL} instruction that not only pops the correct number of control flow labels, but also takes care to upgrade the control flow stack for the current function.
Upgrading the control flow stack in this manner upgrades all the values present on the operand stack at the time the interruption in control flow occurred.
Any later computations performed by the function are then considered to be influenced by the security context in effect at the time the \texttt{break} or \texttt{continue} statement was encountered.

\subsection{Exceptions}
\label{sec:exceptions}
Exceptions represent a substantial challenge to information flow security, because a \texttt{throw} permits any called function to create an early return that crosses multiple function boundaries.
To complicate security issues further, JavaScript supports the \texttt{try}, \texttt{catch}, \texttt{finally} triplet of keywords.

The exception handling region of the \texttt{try}-block begins with a \texttt{DUP\_CFLABEL} instruction.
As a conservative precaution, when the interpreter encounters a \texttt{throw} statement, it takes care to first cloak the exception object that is to be returned to the exception handler using the current pc-label.
Once the interpreter finds the appropriate handler, it pops all activation frames within the call chain.
Control flow then transfers to the corresponding \texttt{catch}-block where a \texttt{POPJ\_CFLABEL} instruction upgrades the entire control flow stack of the handling function using the label taken from the exception object.
Taking this action prevents implicit information leaks that might occur due to exiting the \texttt{try}-block early.
Such leaks are analogous to the \texttt{break} and \texttt{continue} (see Section~\ref{sec:break-and-continue}).

The \texttt{finally}-block always executes using the current pc-label, which is provided either by finishing the \texttt{try}-block or from catching an exception and executing the \texttt{catch}-block.

\subsection{Function calls}
\label{sec:function-calls}
We found it unnecessary to introduce additional instructions to handle function calls.
Instead, we instrument the existing routine for a function call to lookup the label of a function at call time.
When a function call occurs, our system first duplicates the top of control flow stack then joins it with the label of that function.
This action makes all operations which occur within the body of that function safe.
The location of this label breaks down into three cases:
\begin{description}
 \item [The function is provided by a script.]
  When the host environment hands a script to the JavaScript VM, it has the option of labeling that script with a security principal.
  In this case, the function lookup process is responsible for retrieving the label of the function.
 \item [The function is an uncloaked first class value.]
  The current program counter implicitly labels anonymous functions at the time of their creation.
  As a result, the label of the function is always lower than that of the pc-label, and it is safe to call directly.
 \item [The function is a cloaked first class value.]
  If a sensitive control flow computation resulted in the return of a function, the return specification (see Section~\ref{sec:returns}) cloaks the function.
  In this case, the VM retrieves label of the function from its wrapper before the call.
\end{description}

\subsection{Returns}
\label{sec:returns}
A returned value needs to carry information indicating the security context under which the value was produced.
Our system achieves this by explicitly cloaking the return value with the current pc-label using \texttt{CLOAK} instruction.
%\todo{more? if value is already cloaked...}

\subsection{Eval}
\label{sec:eval}
Our system treats \texttt{eval} similar to other function calls.
In this case, the parameter string passed into the \texttt{eval} provides the label for the new execution context.
A call to \texttt{eval} first compiles this string into an instruction stream.
As a result of passing through the parser, this stream contains all of the security instructions as would a normal script.

%The \texttt{eval} frame performs variable access using dynamic lookup.
%The bytecodes emitted for these lookups are the same as the bytecodes emitted for a parent lexical scope (equiv. to global) accesses.
%Our system already mediates the modification of such variables using the \textit{no sensitive upgrade} check (see Section~\ref{sec:sensitive-upgrade-check}), so no extra precautions are necessary for securing the \texttt{eval} construct.

%===============================================================================

\section{Evaluation}
\label{sec:evaluation}

%
%To evaluate our claims we modified SpiderMonkey, the JavaScript VM used in the Mozilla Firefox browser.
%We examine the bytecode overhead introduced by the inclusion of the secure bytecodes, as well as its affect on usability.
%Based on our results from browsing real websites, we give some insight into the architectural changes website authors might have to make to support information flow security.

Our implementation focuses mainly on the creation of a fully functional and correct information flow tracking system.
We would like to note that, to the best of our knowledge, no standard set of tests for information flow frameworks currently exists.
Given this situation, we use the SunSpider~\cite{sunspider} benchmark suite to assess the overhead which our framework introduces.
Despite the fact that it does not faithfully represent real-world JavaScript~\cite{jsmeter}, we choose this suite because its status as the standard benchmark suite for JavaScript makes it suitable for comparisons to other work.

\subsection{Growth of Instruction Stream}
%\begin{figure*}[ht]
%  \centerline{\includegraphics[width=18cm,keepaspectratio=true]{graphics/evaluation_bytecodes.pdf}}
%  \caption{Bytecodes emitted for SunSpider benchmark}
%  \label{fig:eval_bytecodes}
%\end{figure*}

%Figure~\ref{fig:eval_bytecodes} shows the number of emitted bytecodes for each test in the SunSpider benchmark.

To examine impact that introducing the new security instructions has on the memory requirements of SpiderMonkey's internal bytecode representation, we measure the change in size of the emitted instruction stream, for each test in the SunSpider~0.9.1 benchmark suite.
Despite the fact that SunSpider runs only with only a single security principal, we find it useful for measuring the overhead incurred by introducing additional security instructions to well-known algorithms.

% secure = [178, 15, 158, 25, 48, 38, 25, 8, 14, 3, 20, 29, 181, 71, 87, 140, 183, 18, 6, 28, 4, 36, 32, 129, 83, 32]
% total = [ 2238, 173, 2446, 213, 334, 668, 164, 94, 81, 26, 167, 154, 2957, 1753, 930, 1313, 1556, 238, 195, 278, 144, 794, 325, 569, 592, 360]
% data = zip(secure, total)
% d = [ float(x[0]) / (x[1] - x[0]) for x in data ]
% d[8], d[11], d[23], (sum(d)/len(d))
We observe that tests such as \textit{string-tagcloud}, \textit{bitops-bit-in-byte} and \textit{controlflow-recursive}, which contain very ``branchy'' code with respect to their short length, incur the highest overhead; 29.3\%, 21.0\%, and 23.2\% respectively.
Control flow constructs feature much less prominently in the other SunSpider tests, so the introduction of instructions which track branches and merges incur a much lower overhead.
Inserting our security instructions into the instruction stream never causes it to grow by more than 30\%, and maintains an average overhead of 11.6\% growth.

We therefore feel it important to note that many optimization opportunities for reducing the overhead of our techniques remain available for future work.
For example, the overhead of introducing these instructions can likely be optimized away by using a runtime analysis to type-specialize over the labels on arguments and functions.
%Such techniques are very common among interpreter-based virtual machines to optimize ordinary operations.
%For the future however, we plan to follow the quickening approach so that our framework starts executing a bytecode-stream that includes the secure instructions and can jump to an unmodified bytecode-stream whenever possible.

\subsection{Effect on Performance}
\label{sec:evaluation-performance}

We executed the SunSpider~0.9.1 benchmark suite on a Quad Core Intel Xeon~5140 running at 2.33~GHz with 32GiB~RAM running Linux kernel~2.6.3.2.
To achieve a stable basis of comparison, we execute each test in the suite 100 times, and take the average over all runs.

Our modified version of SpiderMonkey requires 140 seconds to execute the entire benchmark, while an unmodified version requires only 22.5 seconds.
We compile both versions using the same flags.
This test demonstrates an overhead of approximately 6x, primarily due to the introduction of explicitly cloaked values at return sites.

Since each cloak is itself a JavaScript object, our system generates a much larger number of objects than an unmodified system, and thus spends more time doing object allocation and garbage collection.
Currently, SpiderMonkey uses an unsophisticated mark-and-sweep garbage collector and does not employ generational collection.
Recent improvements to the garbage collector~\cite{wagner2011} have already demonstrated a remarkable increase in collection speed, which will likely benefit our implementation.
We also believe that we can reduce the number of cloak objects and improve our implementation by using techniques such as sparse labeling~\cite{1554353,1814220}.
%Specifically, we expect to see a large performance improvement by placing objects into a labeled compartment within the garbage collector.

%We also visited several websites, to evaluate this performance overhead as it would appear to a casual user.
%When browsing the web with our system, we did not notice any slowdown for ordinary pages.
%However, we did notice a slight response delay when interacting with JavaScript heavy applications (such as GMail).

\subsection{Violations Issued when Browsing the Web}
We implement enough of the security framework that we can compile and run the Firefox browser.
Although we modify the SpiderMonkey interpreter to track information flow, we found that Firefox uses JavaScript internally for a large number of subsystems (including the user interface).
Despite the potential for covert channels, we chose to whitelist the classes involved in these subsystems and consider them a portion of the trusted code base.


We also used our modified version of Firefox to visit the top 10 sites in the \textit{Alexa's Top Sites}~\cite{alexa} listing.
During this test, our system detects a large number of information flow violations.
Figure~\ref{fig:eval_falsepositives} highlights the total number of unique violations issued for each of these pages.
Manual investigation reveals that the vast majority of these sites load images and other resources from a separate server.
Because sites request these resources from a domain different than the site itself, our system triggers an alarm due to the interaction of DOM objects having separate labels.
For our approach to be adopted in a working environment, web site authors clearly need a policy specification framework (further discussed in Section~\ref{sec:policy_specification}), so that they can express their site's trust in a content distribution server.

\todo[inline]{Figure of false positives}
\begin{comment}
\begin{figure}[htp]
  \centerline{\includegraphics[width=12cm,keepaspectratio=true]{graphics/evaluation_falsepositives.pdf}}
  \caption{Information flow alarms triggered when browsing Alexa's Top Sites in the United States~\cite{alexa}.}
  \label{fig:eval_falsepositives}
\end{figure}
\end{comment}

\subsection{Verification}

In addition to the evaluations presented above, we also implement 173 private test cases to ensure that we generate the correct labels for each of the control flow structures mentioned in Section~\ref{sec:program-control-structures}.
Our system is correctly able to identify both explicit and implicit information flows represented in this test suite, including the two examples presented in Figure~\ref{fig:threat_if} and Figure~\ref{fig:threat_for}.
Although this effort does not substitute for a proof of correctness, it does give us confidence that our implementation faithfully follows the approach we have outlined.

Our system also runs an abstract interpreter that verifies the control flow stack height at every instruction of a method.
This analysis covers \emph{all} possible executions paths for every method parsed, and ensures that we never introduce security instructions that might cause a runtime misalignment of the control flow stack.
We are able to run this verification over all of the more than 2,000 tests in the SpiderMonkey testing suite, which Mozilla uses to detect regressions for every code change.

%\todo{address reviewer comment: Performance overhead is reasonable, some benchmarks using real-world applications would be useful to estimate the actual impact on a system}

%===============================================================================


\todo[inline]{pull citations from acsac-tr}

\todo[inline]{remove the word 'bytecode' and 'opcode', replace with instruction}
\todo[inline]{remove security instruction, replace with control flow instruction}
