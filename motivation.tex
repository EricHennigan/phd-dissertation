
\chapter{Motivation}
\label{ch:motivation}

Many industrial and commercial institutions employ dynamically typed languages in systems which process sensitive information.
Programmers often find the high-level nature of these languages convenient for rapidly developing prototypes that meet the business need for quick deployment.
For example, Web applications have de facto standardized on JavaScript for client-side logic, which includes secure authentication.
In spite of the increasing demand for secure processing and communications, the semantics of dynamically typed languages, such as JavaScript, interferes static analysis techniques which could be used to verify application security~\cite{static-typing}.

In today's architecture, the web server delivers HTML and JavaScript to the client as an untyped string.
Web~2.0 applications contain server-side processing that injects user generated content and syndicated advertisements into pages assembled on demand.
Popular code frameworks which perform this composition are not themselves statically typed (as in the case of PHP, Ruby on Rails, Django).
The practice of including foreign content as an unsafe string provides attackers the opportunity to inject malicious code into a web page, in an attack known as Cross-Site Scripting (XSS).
The prevalence, persistence, scope, and danger~\cite{whitehat, mitre} of such attacks has given XSS the infamous moniker: ``buffer overflow of the web.''

\section{The Threat of Code Injection}

The most pernicious type of XSS attack allows the injected script to persist between browsing sessions, by residing in a server-side data store.
The canonical example of a \term{persistent} XSS attack involves a message board or web forum that allows the posting of user-generated content.
The site stores this content in a server-side database, so that it may be retrieved for viewing by other visitors.
When a malicious user bypasses textual filters and injects JavaScript into a forum posting, the site saves the script and inserts it into all pages that contain the post.
To become a victim, an innocent user needs only to view a page with the malicious post.

Many of the victims viewing the forum are simultaneously logged in to other more sensitive services (such as email, shopping, banking or brokerage accounts).
For most users, the web browser stores more personally identifying information about individual habits and interests than any other single application.
Additionally, web browsers also conveniently store login credentials for banking sites, webmail services, and many shopping sites, as well as form information containing your real name, address, phone number, and credit card numbers.
The potential for harvesting of sensitive user credentials via a forum post with malicious JavaScript presents a serious threat.

\section{Difficulties of Filtering User Input}

Currently, the most promoted mechanism for preventing script injection attacks is to filter out HTML and JavaScript code from user input.
Though user input filtering forms the first line of security defense of any website, not every malicious script can be deterministically identified.
Web frameworks which lack strong static typing further compound the problem because they cannot provably verify that all strings pass through a filter function before placement into the resulting page.

\subsection{HTML Identification}
\label{subsec:html-encoding}

The Web is constructed from various interrelated technologies: URLs for resource requests, HTML for page layout, CSS for content layout, JavaScript for page code, the JavaScript Object Notation (JSON) for object description, XML for data description, etc.
Each of these technologies has its own specification and set of allowed characters.
So many different character encodings exist that developers find it very difficult to track in what context a user-supplied string might appear on a page and how a browser interpret the string in that context.

These many different encodings allow an attacker to formulate strings that are acceptable in one context, but nefarious in another.
For example, characters such as `\texttt{<}', which has special meaning in the HTML specification, can be escaped in many different ways (see Table~\ref{tab:html-encoding}).
Historic design philosophies such as `being liberal in what you accept from others'~\cite{rfc761}, have coerced browsers into allowing whitespace (\texttt{<scr ipt>}) or mixed-case (\texttt{<ScrIpT>}) when matching HTML tags.
The acceptance of tags formatted in unconventional manners further contributes to the difficulty of identifying potentially malicious inputs.

\begin{table}[ht]
\centering
\begin{tabular}{l|ccccc}
 \textbf{Encoding Type} & \multicolumn{4}{c}{\textbf{Encoded variant of `\texttt{<}'}} \\
 \hline
 URL Encoding           & \texttt{\%3C} &&&\\
 HTML Entity            & \texttt{\&lt;} & \texttt{\&lt} & \texttt{\&LT;} & \texttt{\&LT} \\
 Decimal Encoding       & \texttt{\&\#60;} & \texttt{\&\#060;} & \texttt{\&\#0060;} & \texttt{\ldots} \\
 Hex Encoding           & \texttt{\&\#x3c;} & \texttt{\&\#x03c;} & \texttt{\&\#X3c} & \texttt{\ldots} \\
 Unicode                & \texttt{\textbackslash u003c} &&&\\
\end{tabular}
\caption{Examples of Character Encoding~\cite{secubat}.}
\label{tab:html-encoding}
\end{table}

\subsection{Script Identification}

Examples from RSnake's XSS Cheat Sheet~\cite{xsscheatsheet}\footnote{The XSS Cheat Sheet also provides a handy reference of malicious scripts that can be used to test user input filters} exhibit many ways in which a script can be encoded to bypass user input filters.
As a second line of defence, a web page can employ syntax filters and program analysis to restrict the running of malicious scripts~\cite{browsershield, corescript, webssari}.

Just as HTML has many different character encodings, JavaScript provides several syntaxes for accessing an object's properties.
For example, each of the following three lines create a dialog box with the contents of a page's cookie:

\begin{alltt}
alert(document.cookie)
alert(document['cookie'])
with(document) alert(cookie)
\end{alltt}

This multiplicity interferes with routines that attempt to identify malicious code when filtering user input.
To provide a clear demonstration of difficulties encountered by input filtering, Hasegawa~\cite{xssfilters} manufactured the following JavaScript snippet that calls \texttt{alert(1)}, yet contains no alphanumeric characters:

\begin{alltt}
($=[$=[]][(__=!$+$)[_=-~-~-~$]+({}+$)[_/_]+($$=($_=!''+$)
[_/_]+$_[+$])])()[__[_/_]+__[_+~$]+$_[_]+$$](_/_)
\end{alltt}

\section{The Last Line of Defence}

This work assumes that the web application's filter defenses are incomplete and non-exhaustive.
The attacker is able to bypass the filters and store malicious JavaScript in the web site's database.
The server code responsible for assembling pages trusts the database content and includes the code in pages viewable by innocent users of the application.

\begin{comment}
The persistent XSS attack just given illustrates two fundamental principals of web security: (1) user generated content cannot be trusted, and (2) data stored on your own servers should not necessarily be trusted.

The second principal is most concerning, and the one we wish to combat.
Within the browser, the execution semantics of JavaScript,
\end{comment}

Persistent XSS attacks have in the past shut down popular internet services.
For example, in 2005, the `Samy' worm~\cite{samy} set a world record for viral spreading and forced MySpace to suspend service in order to purge the work from their database.
In 2007, the RightMedia Trojan propagated itself via malicious banner advertisements delivered through a trusted syndication network.
These ads appeared on popular sites such as Yahoo! Photobucket and MySpace.
In 2007, Nduja~\cite{nduja} authored a ``Cross Webmail Worm'' that was able to propagate itself across four webmail providers among the most popular in Italy.
On each of the four webmail providers separate functionality for both infection and propagation needed to be developed, making this worm the first to spread across different web applications.
Though dated, these examples are notable not only for setting records, but also for demonstrating the relative ease with which a clever user can control the actions taken by millions of other browsers.
The architectural problems underlying these attacks have not been solved.

\subsection{Preventing Malicious Action}

\begin{comment}
\todo{categorize XSS?}

\begin{itemize}
    \item say that js handles sensitive info
    \item delineate injection
    \item categorize xss
    \item pound-include problem
    \item scare-monger prevalence as `buffer overflow'
    \item sandbox + same origin policy
    \item have to setup an attacker model?
\end{itemize}

\end{comment}

Rather than rely entirely on filtration that attempts to identify and reject attacker supplied code, this work allows the malicious code to execute under new semantics which track the information dependence of runtime values.
We assume that without generating external signals (network traffic) the execution of code within the browser sandbox results in no harm.
This work implements \FlowCore, interpreter-level information flow framework written for WebKit's JavaScriptCore virtual machine.

\FlowCore\ introduces a general set of information flow bytecodes (Section~\ref{sec:bytecodes}) applicable to any interpreter for a dynamically typed language.
The bytecodes provide runtime instructions for manipulating information flow data structures (Section~\ref{sec:f-stack}) for each of the control flow structures (Section~\ref{sec:control-flow}) in the JavaScript language.
Additionally, the VM tags every JavaScript value with a label (Section~\ref{sec:tagging}) that records the principals which have influenced the value.
A network monitor within the web browser prevents malicious program behavior at runtime by enforcing a policy shipped with the web page (Section~\ref{sec:policy}).

\begin{comment}
on using information flow techniques which can detect and prevent malicious behavior of executing programs.

- change language semantics
- augment memory model with labels
- detect and intercept xss
- prevent information leakage

Combine earlier approaches into a universal and comprehensive framework
- decentralized labeling
- support more flexible policies
- hybrid static/dynamic analysis
- dynamically track information flow
- pervasively works at the lowest layer
\end{comment}
