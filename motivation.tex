\begin{comment}
\todo{categorize XSS?}
\begin{itemize}
    \item say that js handles sensitive info
    \item delineate injection
    \item categorize xss
    \item pound-include problem
    \item scare-monger prevalence as `buffer overflow'
    \item sandbox + same origin policy
    \item have to setup an attacker model?
\end{itemize}
\end{comment}

\chapter{Motivation}
\label{ch:motivation}

Systems that process sensitive information often use dynamically typed languages.
The high-level nature of these languages makes them highly convenient for rapidly developing prototypes which meet business needs for quick deployment.
For example, Web applications have de facto standardized on JavaScript for client-side logic, which includes secure authentication.

The World Wide Web has grown explosively and has evolved from a modest collection of static pages to a global network of dynamic and interactive content.
As critical business functions become increasingly Web-based, the amount of sensitive information traversing the Web increased correspondingly.
Personal information, used in sensitive online transactions (e.g., banking, tax filing, shopping, etc.), now possesses a black-market value, which continues to draw the attention of criminals and fraudsters.
Because early designers of the Web did not consider these uses, nor did they have an explicit focus on security, the resulting architecture contains inherent weaknesses that permit criminals to steal credit card numbers, identification credentials, and other personal information.
The dynamic nature of JavaScript reduces the effectiveness of static analysis techniques that would otherwise be used to verify web application security~\cite{robertson.vigna+09}.

Web servers deliver HTML and JavaScript to a client browser as an untyped string.
Web~2.0 applications contain server-side processing that injects user generated content and syndicated advertisements into pages assembled on demand.
Popular server-side code frameworks that perform this composition do not themselves have static typing (as in the case of PHP, Ruby on Rails, and Django).
The practice of including foreign content as an unsafe string provides attackers the opportunity to inject malicious code into a web page, in an attack known as Cross-Site Scripting (XSS).
The prevalence, persistence, scope, and danger~\cite{whitehat, cwe} of such attacks has given XSS the infamous moniker: ``buffer overflow of the web.''

On the client-side, the browser security model possesses weaknesses that integrate HTML and executable JavaScript code into a single execution context, despite possibly separate origins.
Even though this dynamic loading of code functions as a powerful feature enriching web applications with third party code, it also opens the door for attackers to perform arbitrary actions behind the scenes in a users browser.
Without any observable difference in runtime behavior such spurious scripts can exfiltrate sensitive user information, like e.g., authentication credentials or credit card numbers.
A malevolent script has the power to log all keystrokes on a user's keyboard and the ability to traverse the Document Object Model (DOM)~\cite{dom} and exfiltrate all information displayed on the user's screen.

Vulnerability studies consistently rank code injection as the most prevalent type of attack on web applications.
The Open Web Application Security Project\cite{owasp} ranks injection as the biggest threat for its easy exploitability and severe security impact, recommending a parameterized API, taking extra care to correctly handle the escape syntax of any interpreters, and the use of positive ``white list'' input validation routines.
They same organization also ranks XSS as the third biggest threat for its average exploitability, very widespread prevelance and moderate security impact, recommending proper escaping of all untrusted data sources, positive ``white list'' input validation, auto-sanitization libraries for rich content, and the use of a W3C Content Security Policy.
The Common Weakness Enumeration published by MITRE~\cite{cwe} also ranks SQL and OS Command injection in first and second positions, respectively, and XSS in fourth.
The report recommends using libraries with safe APIs, identifying and sanitizing all inputs on both the client and the server, creating static mappings that reference known inputs, and enforcing a separation between code and data.

Nikiforakis~et~al.~\cite{nikiforakis.etal+12} show that 88.45\% of web sites include at least one remote JS library and highlight the potential of such included scripts to perform malicious actions without attracting attention by neither web developers nor end users.
An empirical study~\cite{jang.etal+10} on privacy violating information flows confirms the ubiquity of sensitive user data exfiltration currently practiced on the Internet.

As of today, several approaches~\cite{vogt.etal+07,just.etal+11,groef.etal+12,kerschbaumer.etal+13} have shown that information flow tracking can overcome the shortcomings of the SOP and successfully counter XSS-based information exfiltration attacks.
Even though these dynamic tracking enhancements provide the desired security, all of the presented approaches suffer from the drawback of incurring drastic performance penalties of at least~1$\times$.
Note, that all presented approaches integrate the tracking logic in the JavaScript interpreter which itself commonly performs 4$\times$~to~5$\times$ worse than code generated by state of the art just-in-time (JIT) compilers.

We identify the performance penalty as the major deployment obstacle for adopting one of the presented information flow tracking frameworks and integrate the dynamic tracking of information using a JIT compiler in a JavaScript engine.
This thesis outlines the problematic situation of executing scripts from different origins in the same execution context within a user's browser and makes the following contributions:

\begin{itemize}

\item{To the best of our knowledge we are the first to integrate dynamic information flow tracking in a JIT compiler for a dynamically typed programming language.}

\item{A set of instructions that enables information flow tracking at the bytecode-level, allowing fast transitions between interpreter and JIT-compiled code.}

\item{A first-class mechanism for labeling objects, allowing website authors to tag data considered sensitive to their application and implement label-specific network traffic policies.}

\item{The data structures and implementation techniques that enabled JIT-compilation performance improvements.}

\item{We evaluate the system on two dimensions:}

\begin{itemize}
\item{\textbf{Efficiency:}
We show that our JIT-generated code for information flow tracking introduces, on average, 30\% overhead on established JavaScript benchmark suites;  SunSpider~\cite{sunspider}, V8~\cite{v8}, and Kraken~\cite{kraken}.

\item{\textbf{Compliance:}
We verify that our JIT compiler performs accurate information flow tracking by providing a test suite consisting of more than 200 test cases.}

%\item{\textbf{Real World Applicability:}
%We conduct a web crawler, and show that our system detects real world information flow violations, by randomly visiting 500 web pages of the Alexa~Top~One~Million~\cite{alexa} web pages in the world.}
}
\end{itemize}

\end{itemize}

\section{Executive Summary}

%This work describes \FlowCore, an interpreter-level information flow framework written for WebKit's JavaScriptCore virtual machine, and \JitFlow, a JIT-compiled improvement that provides increased performance with respect to the interpreter.
%Both of these systems tag program values and the program counter with labels that convey data ownership by one or more security principals.
%These labels propagate during program execution, and a network monitor implements a security policy by inspecting the label attached data in a network request.

%To clarify the information flow tracking capabilities of \FlowCore\ and \JitFlow, \autoref{ch:terminology} establishes terminology for the different levels of information flow.
%\autoref{ch:system-design} outlines different implementation strategies and their effects on performance, implementation difficulty, and security label semantics.
%\autoref{ch:label-propagation} describes the underlying data structures that support the label framework.
%As explained in \autoref{ch:instructions}, tracking information dependence through control-flow requires instrumentation of new instructions that manipulate the information flow data structures.
%\autoref{ch:label-tracking} specifies the label propagation strategy used by the framework and illustrates placement of the new instructions through code examples.
%
%To implement unit tests of the label propagation logic, an extension of the browser-hosted JavaScript environment exposes the security labels as first-class program objects (\autoref{ch:first-class-labels}).
%The new instructions provide a versatile architecture that permitted a rapid update of the label propagation system to support JIT compilation (\autoref{ch:jitflow}).
%\autoref{ch:conclusion} summarizes the work involved and lessons learned.

\autoref{ch:attack} gives an overview of web architecture and its negative effects on security.
The structure of HTML, dynamic typing of JavaScript, and the textual inclusion of third-party code and data, each contribute to an infrastructure that puts casual web user's data at risk for surreptitious information theft through a code injection attack known as Cross-Site Scripting.
The widespread practice of dynamically creating strings later treated as HTML and JavaScript and loaded on-demand requires delaying security analysis until runtime.

\autoref{ch:defense} \todo[inline]{supply description}
To detect and mitigate these attacks, we propose using dynamic (runtime) information flow to track data propagation in a JavaScript VM, and call the resulting interpreter \FlowCore.

\autoref{ch:related-work} \todo[inline]{Needs description. How much can be done concurrent with terminology section?}
% \todo[inline]{related work is terrible, missing references, repeated paragraphs, mentions firefox}
%\autoref{ch:related-work} provides the research setting in which the \FlowCore\ implementation took place.
%It summarizes prior efforts to establish guidelines and best practices that have been shown to work, and highlights differences that make previous approaches inapplicable to JavaScript.
%The approaches taken by other researchers 
%It contrasts the approach taken by \FlowCore\ with approaches taken by other researchers.
%The implementation 

\autoref{ch:terminology} establishes and names different levels of information flow tracking detail.
The categorization splits information flows between dataflow and control-flow based leaks and correlates each with a required minimum level of program analysis.
Developing this categorization permits a clear definition of the information flow tracking capabilities \FlowCore.
Through dynamic labeling of the program counter, \FlowCore\ can track up to active implicit information flows.

\autoref{ch:system-design} outlines the different implementations \FlowCore\ might have used.
It contrasts the implementation of labels between a fat value approach and a security wrapper approach.
Each approach is assessed based on the provided labelling semantics, the difficulty of implementation,the ease of label access, and the impacts on the garbage collector and other runtime systems.
Based on the outcome of this analysis, \FlowCore\ chooses to implement the fat values value approach for labelling because it (1) provides a reference semantics, (2) does not impact the garbage collector, and (3) easily labels both primitives and object references.

\autoref{ch:label-propagation} describes implementation details that support labeling values within the JavaScript VM.
\FlowCore\ adds a \FlowLabelRegistry\ that stores a lattice over security principals.
The host web browser can dynamically create a security principal for each web domain, accommodating the common practice of loading code and resources from many different domains in one page of a web application.
Each element of the lattice forms a bit-vector that maps to a label.
\FlowCore\ extends the internal representation of JavaScript values to include space for a label.
Propagating information influence from control flow predicates to instructions within the code branch requires the addition of a label stack that tracks changes to the label on the program counter.
As a JavaScript program executes, the labels attached to program values monotonically rise through the lattice, leading to label creep.

\autoref{ch:instructions} introduces three new instructions that maintain the control flow label stack.
A parse-time analysis instruments these instructions into the instruction stream.
In debug mode, an abstract interpretation that explores all control flow paths of the instruction stream ensures correct instrumentation for each method.
During execution, the instructions push, pop, and upgrade the stack according to branches, joins, and loops in the control flow.
The development of these instructions enables a transition in implementation from the interpreter (\FlowCore) to the JIT compiler (\JitFlow).
The instructions are generic enough that they support the grafting of information flow tracking into both register-based (WebKit) and stack-based (Firefox) VM implementations.

\autoref{ch:label-tracking} outlines the semantics and implementation of label propagation that occurs in \FlowCore.
It specifies the label inputs and outputs for mathematical, comparison, and bitwise operations, that occur during data flow operations.
Through illustrative examples, it gives the implementation of label prapagation for JavaScript control-flow features, \code{for} and \code{while} loops, including \code{break} and \code{continue} statements.
These examples demonstrate the placement of the control flow stack instructions.
It describes the label propagation rules for function calls, including the implementation details that affect the control flow stock during function prologue and epilogue.

\autoref{ch:first-class-labels} describes an extension of the labeling framework that gives JavaScript developers access to the labeling system.
\FlowCore\ reflects labels stored in the \FlowLabelRegistry\ as first-class objects into the host environment.
The new \FlowLabelObject\ prototype class enables the creation, comparison, composition of labels, while a new \mlabelof\ keyword enables inspection of labeled variables.
When used with a modified web browser, a network monitor hook allows the writing of a security policy within JavaScript.
Using first-class labels developers can tag specific values with their own labels, decreasing the sources of label creep and using the propagation and inference rules as a security debugging system.
Additionally, the first-class label feature proved invaluable for implementing unittests of the propagation rules and buttressing confidence in the correctness of the implementation.

\autoref{ch:jitflow} demonstrates that the framework developed for the \FlowCore\ interpreter readily supports transition to a JIT compiler.
The new tracking engine, \JitFlow, implements the control-flow stack instructions in assembly, caches the top of the control flow stack in a virtual register, and pre-allocates space for the control flow stack itself.
These optimization lead to a vast performance improvement, that approximately halves the execution time of the flow tracking interpreter.
